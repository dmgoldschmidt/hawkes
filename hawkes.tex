\documentclass[12pt,leqno]{article}
\include{article_defs}
\title{A Primer on the Hawkes Process}
\author{David M. Goldschmidt}
%\oddsidemargin 10 pt \evensidemargin 10 pt \marginparwidth 0.75 in \textwidth
%6.0 true in \topmargin -40 pt \textheight 8.8 true in 
%\usepackage{fancyhdr}
%\pagestyle{fancy}
%\lfoot{}
%\rfoot{\thepage}
\begin{document}
%\renewcommand{\footrulewidth}{0.4pt}
\newcommand{\p}{\ensuremath{u}}
\newcommand{\VV}{V}
\maketitle


\section{Introduction}
In this expository paper, we review the basics of the Hawkes process, step by step.

\section{The Exponential Probability Density}
The exponential density with rate $\lambda$ is defined on the non-negative real line and has the probability
density function
\begin{equation}
  f_{\lambda}(t) := {\lambda}e^{-\lambda{t}}.
\end{equation}
The reason the parameter $\lambda$ is called the {\em rate} is that the expected value of $f_{\lambda}$ can be found by
integrating by parts:
\begin{align*}
E_{\lambda} &= \lambda\left(\int_0^{\infty}te^{-\lambda{t}}dt\right) = \lambda\left(\lambda^{-1}{t}e^{-\lambda{t}}\biggr\rvert_0^{\infty}
-\lambda^{-1}\int_0^{\infty}e^{-\lambda{t}}dt\right)\\
&= \lambda(\lambda^{-2}) = \lambda^{-1}.
\end{align*}
So if a sequence of events is arriving with the {\em inter-arrival time} $\tau$ between successive events
distributed as $f_{\lambda}(\tau)$, then the expected value of $\tau$ is $\lambda^{-1}$ units of time per
event, which means that the mean rate of arrivals is $\lambda$ events per unit of time.

We can sample from this density as follows.  Let
\begin{equation} \label{y=F(t)}
y = F_{\lambda}(t) := \int_0^{t}f_{\lambda}(\tau)d\tau = -e^{-\lambda\tau}\biggr\rvert_0^t = 1 - e^{-\lambda{t}}
\end{equation}
be the cumulative distribution function of $f_{\lambda}$. Then $y$ is a random variable on $[0,1]$ which we claim
is uniformly distributed.  Namely, let $y_0 = F_{\lambda}(t_0)$, then the probability that $y$ lies between $y_0$ and
$y_0+dy$ is the same as the probability that $t$ lies between $t_0$ and $t_0+dt$, which is the area of the
infinitesmal rectangle $f_{\lambda}(t_0)dt$  under $f_{\lambda}(t)$.  Since $dy = F'_{\lambda}(t)dt = f_{\lambda}(t)dt$,
we conclude that $y$ is uniformly distributed as claimed.\footnote{This argument is completely general.  It has nothing to do with the details of the density $f_{\lambda}$.}

Now from \eqref{y=F(t)} we obtain
\begin{equation}\label{t=Finv(y)}
  e^{-\lambda{t}} = 1 -y,\quad\text{hence}\quad t = -\frac{\log(1-y)}{\lambda}.
\end{equation}
Thus, to sample $t$ according to the density $f_{\lambda}(t)$, we first sample $y$ uniformly and then apply
\eqref{t=Finv(y)}.

\section{The EM algorithm}
In this section we review the application of EM algorithm to mixture densities.

\subsection{The Q-function}\label{Qf} 
\begin{Lem}\label{Qfunct} Suppose that $f_1(x),\dots,f_n(x)$ are positive real-valued functions.
Let $f(x):= \sum_{i=1}^nf_i(x)$, and define 
$$
Q(x',x):= \sum_{i=1}^nf_i(x')\log(f_i(x)).
$$
Then $ f(x) - f(x') \ge Q(x',x) - Q(x',x')$ with equality only if
$Q(x',x) = Q(x',x')$.
\end{Lem}
\begin{proof}
The equation of the tangent line to the graph of $y = \log(x)$ at $x = 1$ is $y = x - 1$.
Since $\log(x)$ is concave down, it follows that 
$\log(x) \le x - 1$, with equality if and only if $x = 1$. Using this, we have
\begin{align*}
Q(x',x) - Q(x',x') &= \sum_{i=1}^nf_i(x')[\log(f_i(x) - \log(f_i(x'))] \\
&= \sum_{i=1}^nf_i(x')\log(f_i(x)/f_i(x')) \\
&\le \sum_{i=1}^nf_i(x')[f_i(x)/f_i(x') - 1] \\
&= \sum_{i=1}^n[f_i(x) - f_i(x')] \\
&= f(x) - f(x'),
\end{align*}
with equality if and only if $f_i(x)/f_i(x') = 1$ for all $i$.
\end{proof}

Of course, the point is that it is often possible to write a (log-)likelihood
function $L(\theta)$ as a sum $L(\theta) = \sum_iL_i(\theta)$ where $L_i(\theta)$ has
a relatively simple form. Then given a set of parameters $\theta'$, we can find a
more likely set $\theta$ by choosing $\theta$ so that $Q(\theta',\theta) >
Q(\theta',\theta')$.  In practice, we usually maximize $Q(\theta',\theta)$ as a function of
$\theta$.  We can then replace $\theta'$ by $\theta$ and iterate until a {\em
  local} maximum of $L(\theta)$ is reached.    This iteration is referred to in the literature
as the EM algorithm.

\subsection{Mixture Densities}\label{MD}
A standard application of the $Q$-function technology is to the maximum
likelihood problem for a mixture density.  Here we have a set of
probability densities $p_1(\eta;y),\dots,p_s(\eta;y)$ where $\eta$
denotes a set of parameters and $y$ is an observation.  The yoga is that there
is a "hidden state" $i\in\{1,2,\dots,s\}$ such that $p_i(\eta;y)$ is the
conditional probability density of $y$ given parameters $\eta$ and state $i$.

Given a vector $\omega:= \{\omega_1,\dots,\omega_n\}$ of state probabilities, we put
$\theta:= (\eta,\omega)$.  The mixture density is then
$$
p(\theta;y) = \sum_{i=1}^s\omega_ip_i(\eta;y).
$$

Now suppose we are given a sequence $\{y_1,\dots,y_N\}$ of observations which we
assume are i.i.d. samples from this density,
and we want choose parameters $\theta$ to maximize the likelihood of the observations.
The likelihood function is  
\begin{equation*} \begin{split} 
L(\theta)
&:= \prod_{n=1}^N\sum_{i=1}^s\omega_ip_i(\eta,y_n) \\ 
&=
\sum_{i_1,\dots,i_N=1}^s\prod_{n=1}^N\omega_{i_n}p_{i_n}(\eta,y_n), \\ 
\end{split} 
\end{equation*} 
and the $Q$-function for this sum is 
$$
Q(\theta',\theta) = \sum_{i_1,\dots,i_N=1}^s\prod_{n=1}^N\omega'_{i_n}p_{i_n}(\eta';y_n)
\sum_{k=1}^N\log(\omega_{i_k}p_{i_k}(\eta;y_k)).
$$
The good news here is that there is a significant simplification of this expression.
Namely, the coefficient of $\log(\omega_mp_m(\eta;y_n))$ in the above sum is given by
\begin{align*}
\omega'_mp_m(\eta';y_n)\sum_{I:i_n = m}\prod_{j\neq n}\omega'_{i_j}p_{i_j}(\eta';y_j )
&=\omega'_mp_m(\eta';y_n)\prod_{j\neq n}\sum_{i=1}^s\omega'_ip_i(\eta';y_j)\\
&= \frac{\omega'_mp_m(\eta';y_n)}{p(\theta';y_n)}\prod_{j=1}^Np(\theta';y_j).
\end{align*}
Note that the quantity 
\begin{equation}\label{omega_mn}
\omega'_{mn} := \frac{\omega'_mp_m(\eta';y_n)}{p(\theta';y_n)}
\end{equation}
is the conditional probability of state $m$ at observation time $n$, given observation $y_n$ 
and parameters $\eta'$.
Thus, dividing by the constant $\prod_jp(\theta',y_j)$ and changing notation, the function to be
maximized is
\begin{equation}\label{Qfunc:0}
\tilde{Q}(\theta',\theta) := \sum_{m=1}^s\sum_{n=1}^N\omega'_{mn}\log(\omega_mp_m(\eta;y_n)),
\end{equation}
where, as above, $\omega'$ denotes the prior value, while $\omega$ and $\eta$ are the unknown parameters
to be solved for which will maximize $\tilde{Q}$.


\section{The Poisson Process}
For our purposes, a Poisson process of rate $\lambda$ will be defined as an increasing sequence of random variables
$ \{0 < T_1 < T_2 < \dots\}$ defined on $(0,\infty)$ such that $\{\tau_i = T_i-T_{i-1}~|~ i = 1,2,\dots\}$ are i.i.d.
exponential of rate $\lambda$. We often think of $T_i$ as the time of occurence of the $i^{th}$
event, where often the event is the arrival of something.  So in the following, we will call the $T_i$
``arrival times'' and the $\tau_i$ ``inter-arrival times''.  The probability density of any finite set of arrival times
$S_n = \{T_1 = t_1 < T_2 = t_2<\dots<T_n = t_n\}$ is easy to compute. Using independence we see that
\begin{equation}\label{arr_seq}
Pd(S_n) = \prod_{i=1}^n\lambda{e}^{-\lambda(t_i - t_{i-1})} = \lambda^ne^{-\lambda\sum_{i = 1}^n(t_i-t_{i-1})} = \lambda^ne^{-\lambda{t_n}}.
\end{equation}
In particular, the density of any particular arrival sequence does not depend on the intermediate arrival times
at all, only on the number of arrivals and the time of the last arrival.

Now, the actual quantity of interest here
is $Pd(T_n = t)$, the probability density that the $n^{th}$ arrival occurs at time $t$.
\begin{Thm}
  Let $\{0 < T_1<\dots<T_n<... \}$ be a Poisson process of rate $\lambda$.  Then
  \begin{equation}\label{erlang}
  Pd(T_n = t) = \frac{\lambda^nt^{n-1}}{(n-1)!}e^{-\lambda{t}}.
  \end{equation}
  \begin{proof}
    Put $\tau_i = t_i - t_{i-1}$ for all $i$ as above. Then the $\tau_i$ are i.i.d. exponential of rate $\lambda$, and
    $t_j = \sum_{i=1}^j \tau_i$ for all $j$.  In particular $t_1 = \tau_1$ so the theorem is trivial for $n = 1$.
    Inductively, assume that it holds for $n-1$. Then $t = t_{n-1} + \tau_n$ is the sum of two independent random
    variables and hence $Pd(T_n = t)$ is the convolution
    \begin{align*}
      Pd(T_n = t) &= Pd(T_{n-1} = s)*({\lambda}e^{-\lambda(t-s)})\\
      &= \lambda\int_{s=0}^t\frac{\lambda^{n-1}s^{n-2}}{(n-2)!}e^{-\lambda{s}}e^{-\lambda(t-s)}ds \\
      &= \frac{\lambda^n}{(n-2)!)}e^{-\lambda{t}}\int_{s=0}^ts^{n-2}ds \\
      &= \frac{\lambda^nt^{n-1}}{(n-1)!}e^{-\lambda{t}}.
    \end{align*}
  \end{proof}
\end{Thm}

By integrating \eqref{erlang} over the interval $0<t<=T$, we obtain
\begin{Cor}
  For a Poisson process of rate $\lambda$ as in the theorem, 
  \begin{equation}\label{poisson}
  Pr(T_n \le T) = \frac{\lambda^nT^{n}}{n!}e^{-\lambda{T}}.
  \end{equation}
  \qed
\end{Cor}

\begin{Cor}\label{Pr(obs|state)}
  For a Poisson process of rate $\lambda$ as in the theorem, the probability density that the $k^{th}$ arrival occurs at
  time $t$ for some $k \le n$ is
  \begin{equation}
    e^{-\lambda{t}}\sum_{k=1}^{n}\frac{(\lambda{t})^k}{k!}.
  \end{equation}
\end{Cor}\qed

\section{The Hawkes Process}
\subsection{A branching Poisson process}
We now introduce dependencies among the arrival times $T_i$, by allowing each arrival event 
to spawn a new Poisson process, which we call a child process as opposed to the original process which we
call the base process.  This leads to a view of the
Hawkes process as a branching Poisson process.  However there is one important difference between the base process and
all the child processes.  Namely, the base process has a constant rate $\lambda_0$ \footnote{This can be generalized to a time-dependent rate, but it is usually not exponentially decaying.} while the child processes
all have exponentially decaying rates.  Denote the base
process by ${\cal{P}}_0$ and  the child process spawned at arrival time
$t_j~(j > 0)$ by ${\cal{P}}_j$.  

There is one further important generalization, namely we may see
different types of events.  Event types are often called ``marks'' in the literature, and a process with different
event types is called a ``marked process''. 
Then our observations are pairs $(t,m)$ meaning that an event of type $m$ has occured at time $t$.  In this
case we denote the mark of event $i$ by $m(i)$, and we allow the exponential decay parameters for child process
${\cal{P}}_j$ to depend on $m(j)$, the mark of the originating event at arrival time $t_j$.  Then for $j>0$,
${\cal{P}}_j$ has rate
\begin{equation}\label{child_rate}
\lambda_j(t) =
\begin{cases}
  \mu_{m(j)}e^{-\nu_{m(j)}(t-t_j)}\quad\text{for $t > t_j$}\\
  0\quad\text{for $t \le t_j$}.
\end{cases}
\end{equation}
We define $\lambda_{ij} := \lambda_j(t_i)$ for $0 <j < i$,
$\lambda_{i0} := \lambda_0$ for all $i$, and $t_{ij} := t_i-t_j$.

\subsection{The Model}
We want to explain an observed arrival stream $\{t_0 = 0, (t_1,m(1),\dots,(t_N,m(N)\}$ as the output
of a Hawkes process.
Thus, if $t_i$ is the arrival of the $k_j^{th}$ event generated by process ${\cal{P}}_j$ for some index $j < i$,
its density is
\begin{equation}\label{p_ij}
  p_{ij} :=  \frac{\lambda_{ij}^{k_j}t_{ij}^{(k_j-1)}}{(k_j-1)!}e^{-\lambda_{ij}t_{ij}}
\end{equation}
by \eqref{erlang}, where $\lambda_{ij}$ and $t_{ij}$ were defined above.

Unfortunately, however, the neither the value of $k_j$ nor even the value of $j$ is  observed, i.e. we don't know
which process ${\cal{P}}_j$ generated the event $T_i = t_i$, nor do we know how many events ${\cal{P}}_j$
generated prior to $t_i$.  Thus, we can't use \eqref{erlang} directly. Instead,
we define a hidden state space $S := \{0,1,2,\dots,N\}$ and assume that $j\in{S}$ has probability $Pr(j) := \omega_j$.
This makes $j$ a random variable with a prior distribution, but we still don't know the value of $k_j$.  So we
will replace $k_j$ by its expected value $\hat{k}_j$, the expected number of events generated by ${\cal{P}}_j$
in the interval $(t_j,t_i]$, which is  obtained by integrating the arrival rate $\lambda_j(t)$ from $t_j$ to $t_i$,
and is easily seen to be given by
$$
\hat{k}_j =
\begin{cases}
  \frac{\mu_{m(j)}}{\nu_{m(j)}}(1-e^{-\nu_{m(j)}t_{ij}})\quad\text{for $j > 0$,} \\
  \lambda_0t_i\quad\text{for $j = 0$}.
\end{cases}
$$
Then our likelihood function is
$$
\prod_{i=1}^NPd(T_i = t_i)
$$
where 
\begin{equation}\label{q_ij}
  Pd(T_i = t) = q_{ij} := \sum_{j = 0}^{i-1}\omega_j\frac{\lambda_{ij}^{\hat{k}_j}t_{ij}^{(\hat{k}_j-1)}}{\Gamma(\hat{k}_j)}
    e^{-\lambda_{ij}t_{ij}},
\end{equation}
where we have replaced $k_j$ in \eqref{p_ij} by $\hat{k}_j$ and $(k_j-1)!$ by $\Gamma(\hat{k_j})$
(because $\hat{k}_j$ is not in general an integer).

\subsection{Optimizing the Parameters}
The observations are $\{(t_0 = 0 < (t_1,m(t_i),\dots(t_N,m(t_N)\}$.  Let $M$ denote the set of distinct marks $m(t_i)$
seen in the data.  Our model parameters are
$$
\theta := \lambda_0, \{\mu_m, \nu_m, \mid m \in M\},~\text{and}~ \{\omega_j\mid 0\le j\le N\},
$$
and the Q-function to be maximized (see \eqref{Qfunc:0}) becomes
$$
\tilde{Q}(\theta) = \sum_{i=1}^N\sum_{j=0}^N\omega_{ij}'\log(\omega_jq_{ij}),
$$
where, following \eqref{omega_mn}, $\omega'_{ij}$ is the (prior) conditional
probability of the event $T_i=t_i$,
given that it was generated by process ${\cal{P}}_j$.  We compute $\omega'_{ij}$ by
treating the density $\omega'_jq'_{ij}$ as a likelihood and normalizing to a distribution:
$$
\omega'_{ij} := \frac{\omega'_jq'_{ij}}{\sum_j\omega'_jq'_{ij}}.
$$

We can now maximize $\tilde{Q}(\theta)$ in closed form by setting partial derivatives with respect to $\theta$ to
zero and solving the resulting equations for the posterior parameters.  We first solve for $\omega_j$, keeping in
mind the constraint
$$
\sum_{j=0}^N\omega_j = 1,
$$
which we will enforce using a Lagrange multiplier $\rho$ as follows:

\begin{align*}
  0 &= \frac{\partial}{\partial\omega_j}\left(\tilde{Q} + \rho(\sum_{j=0}^N\omega_j - 1)\right) \\
  &= \frac{1}\omega_j\sum_{i=1}^N\omega_{ij}' - \rho.
\end{align*}
We conclude that
\begin{equation}\label{omega_j:0}
  \begin{split}
  \rho\omega_j &= \sum_{i=1}^N\omega_{ij}',\quad\text{whence} \\
  \rho\sum_{j=0}^N\omega_j &= \sum_{i=1}^N\sum_{j=0}^N\omega'_{ij}.
  \end{split}
\end{equation}
Since
$$
\sum_{j=0}^N \omega_j = 1 = \sum_{j=0}^N\omega'_{ij},
$$
it follows that $\rho = N$ and thus \eqref{omega_j:0} yields
\begin{equation}\label{omega_j}
  \omega_j = \frac{1}{N}\sum_{i=1}^N\omega'_{ij}.
\end{equation}
To solve for the remaining parameters, we need some intermediate derivatives:

\begin{align*}
  D_{ij} := \frac{\partial{q_{ij}}}{\partial\lambda_{ij}} &= \frac{\partial}{\partial\lambda_{ij}}\left(\lambda_{ij}e^{-t_{ij}(1-\delta_{j0})\lambda_{ij}}\sum_{k=1}^i\frac{(\lambda_{ij}t_{ij})^k}{k!}\right)\\
  &= e^{-t_{ij}(1-\delta_{j0})\lambda_{ij}}\sum_{k=1}^i\frac{(\lambda_{ij}t_{ij})^k}{k!}\\
  &-\lambda_{ij}t_{ij}(1-\delta_{j0})e^{-t_{ij}(1-\delta_{j0})\lambda_{ij}}\sum_{k=1}^i\frac{(\lambda_{ij}t_{ij})^k}{k!}\\
  &+\lambda_{ij}e^{-t_{ij}(1-\delta_{j0})\lambda_{ij}}\sum_{k=1}^i\frac{\lambda_{ij}^{k-1}t_{ij}^k}{(k-1)!}.
\end{align*}

We next solve for $\lambda_0$:
$$
0 = \frac{\partial\tilde{Q}}{\lambda_0}
$$
  
  
  

At first glance, it appears that we might have a large number of states because the number of states $N+1$ is (one) more than the number of observations,  However, note first that $q_{ij} = 0$ for $j \ge i$, and second that
$\omega_{ij}$ decays exponentially in $t_i-t_j$, so in practice we can ignore
states $j << i$.

There is another view of the Hawkes process as an inhomogeneous Poisson process whose intensity changes by positive
jumps at each arrival time.  The two views are mathematically equivalent, but we will take the branching view here.

 

\end{document}



Next, we recall Horner's rule for polynomial evaluation, namely for any positive integer $m$ and any
real number $x$, we define
$$
f_0(m,x) = 1\quad\text{and}\quad f_{k+1}(m,x) = 1+\frac{x}{m-k}f_k(m,x)\quad(0\le k\le m-1).
$$
Then
$$
f_m(m,x) = \sum_{k=0}^m\frac{x^k}{k!}.
$$
Horner's rule is easy to verify by induction, and is more efficient and more numerically stable than the
naive algorithm.
